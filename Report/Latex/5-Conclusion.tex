In conclusion, two alignment algorithms are reported. 

The first algorithm without ORB uses ECC Maximization algorithm to find the parameters for the affine transformation and improve the alignment result overall. The second algorithm uses ORB developed by Open CV to find and match corresponding feature points. The mapping is then used to find the parameters of the homography transformation. This algorithm can perfectly align two maps of similar shapes or structures. However, it only works 50\% of the time. Therefore it can be used as a supplement to the first algorithm to improve the overall alignment result. The number of active pixels is used to quantize how good the alignment is. This is proved to be reasonable with the example given. Search radius and dilation with Gaussian smoothing both significantly affect the second algorithm's performance.

There is still a lot to work on for this project. 

With the alignment algorithm, one can easily take the average and get the ground truth map of each period. With this, one intriguing topic is to compare the number of road pixels in a neighborhood in different periods. Different averaging methods can also be applied. One method is to only average maps published in the same year. Another method is to add maps of the adjacent years but with a lower weight. Search radius is a critical parameter of the second alignment algorithm with ORB. In this report, the discussion and research on the search radius are pretty limited. We initialized the search radius manually and only tested the search radius in a minimal range. The algorithm's success rate might be improved with more in-depth research of the search radius.